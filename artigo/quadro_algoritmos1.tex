\begin{tabular}{|c|c|c|c|}
\hline
\textbf{Algoritmo}          & \textbf{Sequencial} & \textbf{Por Saltos} & \textbf{Binária}                                                                                                                                        \\ \hline
\textbf{Funcionamento}      & \begin{tabular}[c]{@{}c@{}}Compara a chave de busca, elemento a\\ elemento, do início ao fim do arranjo,\\ até encontrá-la ou exaurir o conjunto\end{tabular} & \begin{tabular}[c]{@{}c@{}}Salta até encontrar o bloco que\\ deveria conter a chave de busca,\\ então realiza uma busca sequencial\\ no bloco.\end{tabular} & \begin{tabular}[c]{@{}c@{}}Reiteradamente, particiona o arranjo ao meio,\\ comparando a chave de busca com o elemento\\ central da partição, até encontrá-la ou\\ obter uma partição com um único elemento.\end{tabular} \\ \hline
\textbf{Abordagem}          & Linear & Divisão e conquista & Divisão e conquista                                                                                                                                                  \\ \hline
\textbf{Melhor caso}        & \begin{tabular}[c]{@{}c@{}}O(1)\\ 1º elemento do arranjo\end{tabular}                                                                                                               & \begin{tabular}[c]{@{}c@{}}O(2)\\ 1º elemento do 1º bloco\end{tabular}                                                                                                              & \begin{tabular}[c]{@{}c@{}}O(1)\\ Elemento central do arranjo\end{tabular}                                                                                                                                               \\ \hline
\textbf{Caso médio}         & O(N)                                                                                                                                                        & \protect\footnotemark O($\sqrt{N}$)                                                                                                                                          & O($log_{2} N$)                                                                                                                                                                                                         \\ \hline
\textbf{Pior caso}          & \begin{tabular}[c]{@{}c@{}}O(N)\\ Último elemento do arranjo ou\\ inexistente\end{tabular}                                                                                          & \begin{tabular}[c]{@{}c@{}}\footref{jump_search_simple}O($\sqrt{N}$)\\ Último elemento do último bloco\end{tabular}                                                                                             & \begin{tabular}[c]{@{}c@{}}O($log_{2} N$)\\ Último elemento da última partição\end{tabular}                                                                                                                              \\ \hline
\textbf{Ordenação}          & Opcional                                                                                                                                                    & Obrigatória                                                                                                                                             & Obrigatória                                                                                                                                                                                                            \\ \hline
\textbf{Estrutura de dados} & \begin{tabular}[c]{@{}c@{}}Vetores\\ Listas encadeadas\end{tabular}                                                                                         & Vetores                                                                                                                                                 & Vetores                                                                                                                                                                                                                \\ \hline
\textbf{Linhas Python}      & 5                                                                                                                                                           & 15                                                                                                                                                      & 10                                                                                                                                                                                                                     \\ \hline
\textbf{Observações}        & \begin{tabular}[c]{@{}c@{}}Fácil de compreender e implementar\\ Funciona em diversas estruturas de dados\\ Não requer ordenação prévia\\ Custoso (tempo e comparações)\end{tabular} & \begin{tabular}[c]{@{}c@{}}O mais complexo dos 3 métodos\\ Requer adaptações para encadeamento\\ Necessita de ordenação prévia\\ Mais eficiente que a busca sequencial\end{tabular} & \begin{tabular}[c]{@{}c@{}}Relativamente simples\\ Requer adaptações para encadeamento\\ Necessita de ordenação prévia\\ Mais eficiente dos 3 métodos\end{tabular}                                                       \\ \hline
\end{tabular}

