%%%%%%%%%%%%%%%%%%%%%%%%%%%%%%%%%%%%%%%%%%%%%%%%%%%%%%%%%%%%%%%%%%%%%%
% How to use writeLaTeX: 
%
% You edit the source code here on the left, and the preview on the
% right shows you the result within a few seconds.
%
% Bookmark this page and share the URL with your co-authors. They can
% edit at the same time!
%
% You can upload figures, bibliographies, custom classes and
% styles using the files menu.
%
%%%%%%%%%%%%%%%%%%%%%%%%%%%%%%%%%%%%%%%%%%%%%%%%%%%%%%%%%%%%%%%%%%%%%%

\documentclass[12pt]{article}

\usepackage{sbc-template}

\usepackage{graphicx,url}

\usepackage[brazil]{babel}
\usepackage[utf8]{inputenc}  

\usepackage{adjustbox}
\usepackage{booktabs}
\usepackage{multirow}

\usepackage{algorithm}
\usepackage{algorithmic}

\floatname{algorithm}{Algoritmo}
\renewcommand{\algorithmicrequire}{\textbf{Entrada:}}
\renewcommand{\algorithmicensure}{\textbf{Saída:}}
\renewcommand{\algorithmicif}{\textbf{se}}
\renewcommand{\algorithmicthen}{\textbf{então}}
\renewcommand{\algorithmicend}{\textbf{fim}}
\renewcommand{\algorithmicforall}{\textbf{para todo}}
\renewcommand{\algorithmicfor}{\textbf{para}}
\renewcommand{\algorithmicreturn}{\textbf{retorne}}
\renewcommand{\algorithmicelse}{\textbf{senão}}
\renewcommand{\algorithmicdo}{\textbf{faça}}
\renewcommand{\algorithmicfalse}{\textbf{falso}}
\renewcommand{\algorithmicwhile}{\textbf{enquanto}}
\renewcommand{\algorithmiccomment}[1]{\hspace{2em}// #1}

\sloppy

\title{Estruturas de Dados e Análise de Algoritmos - EDAA\\ Avaliação 1.1 – Algoritmos de Busca – Parte 1}

\author{Rodrigo Schmidt Nurmberg\inst{1}}

\address{Programa de Pós Graduação em Ciência da Computação (PPGCOMP)\\ Universidade Estadual do Oeste do Paraná (UNIOESTE)\\
  Rua Universitária, 2069 Bloco B -- Bairro Universitário -- 85819-110 -- Cascavel -- PR
  \email{rodrigo.nurmberg@unioeste.br}
}

\begin{document} 

\maketitle

\begin{resumo} 
Buscas são operações fundamentais na computação. Existem diversos métodos de busca, a seleção do mais adequado depende da natureza da aplicação. Neste trabalho foram implementados e discutidos 3 métodos clássicos de busca: sequencial, por saltos e binária.
Os métodos foram comparados empiricamente, quanto ao número de comparações e ao tempo de execução, na busca de valores inteiros em arranjos estáticos, com base em dois cenários: aleatório e pior caso. Os resultados mostram que o algoritmo de propósito mais geral, busca sequencial, apesar de apresentar maior tempo médio de execução é mais rápido em buscas únicas que os demais métodos, que demandam ordenação prévia dos dados.
\end{resumo}

\section{Introdução}

Busca e ordenação são operações essenciais na computação, como são utilizadas com frequência, mesmo pequenas otimizações podem gerar grandes impactos, principalmente ao operarem sobre grandes conjuntos de dados.

A operação de busca consiste em determinar a posição ocupada por um elemento (chave de busca) em um conjunto de dados\footnote{Caso o elemento não esteja no conjunto, costuma-se retornar vazio ou falso.}.

Na ordenação, para um dado conjunto de entrada $\langle a_{1}, a_{2}, ..., a_{n} \rangle$, retorna-se uma permutação da sequência de entrada $\langle a^{'}_{1}, a^{'}_{2}, ..., a^{'}_{n} \rangle \vert a^{'}_{i} \leq a^{'}_{i+1} \forall i_{1}^{n-1}$\cite{cormen_algoritmos:_2012}\footnote{Também é possível ordenar os valores de forma decrescente: $\langle a^{'}_{1}, a^{'}_{2}, ..., a^{'}_{n} \rangle \vert a^{'}_{i} \geq a^{'}_{i+1} \forall i_{1}^{n-1}$}.

A escolha do algoritmo mais adequado, depende da natureza de cada aplicação e deve considerar, entre outros fatores, se os dados encontram-se ordenados, a forma como os dados estão armazenados (unidades de acesso aleatório ou sequencial, cada qual com sua velocidade), as estruturas de dados utilizadas (vetores, listas encadeadas, etc), quantidade de itens a ordenar e como estão distribuídos (uniformemente ou não) e também a arquitetura do computador (com diferentes operações e seus custos).

\subsection{Algoritmos de busca} \label{sec:algoritmos}

A seguir, são apresentados os pseudo-códigos dos 3 algoritmos de busca analisados neste trabalho, cujas características são sumarizadas no quadro comparativo da Tabela \ref{quadro:comparativo_buscas}.

Os algoritmos tem como entrada o arranjo $ A^n_{i=1} \langle a_{1}, a_{2}, ..., a_{n} \rangle $ e a chave de busca $ x $, e como saída o índice $ i $ do 1º elemento $ a_{i} \vert a_{i} = x $ ou falso caso $ x \notin A $.

\begin{algorithm}
    \caption{ - Busca sequencial \cite[p. 396]{knuth1998art}}
    \begin{algorithmic}[1]
        \STATE $i \leftarrow 1$
        \WHILE{$ i \leq n $}
            \IF{$ x = A[i]$}
                \RETURN $ i $
            \ENDIF
            \STATE $i \leftarrow i + 1$
        \ENDWHILE
        \RETURN \FALSE
    \end{algorithmic}
    \label{alg:busca_sequencial}
\end{algorithm}
\begin{algorithm}
    \caption{ - Pseudo-código busca por saltos \cite{shneiderman_jump_1978}}
    \begin{algorithmic}[1]
        \STATE $salto \leftarrow \lfloor \sqrt{n} \rfloor$
        \STATE $i \leftarrow 1$
        \STATE $j \leftarrow min(salto, n) + 1$
        \WHILE{$ j \leq n $}
            \IF{$ A[j] \geq x$}
                \WHILE{$ i \leq j $}
                    \IF{$ x = A[i]$}
                        \RETURN $ i $
                    \ENDIF
                    \STATE $i \leftarrow i + 1$
                \ENDWHILE
                \RETURN \FALSE
            \ENDIF
            \STATE $i \leftarrow j$
            \STATE $j \leftarrow min(j + salto, n)$
        \ENDWHILE
        \RETURN \FALSE
    \end{algorithmic}
    \label{alg:busca_saltos}
\end{algorithm}

\begin{algorithm}
    \caption{ - Busca binária \cite[p. 410]{knuth1998art}}
    \begin{algorithmic}[1]
        \STATE $esquerda \leftarrow 1$
        \STATE $direita \leftarrow n + 1$
        \WHILE{$ esquerda \leq direita $}
            \STATE $meio \leftarrow esquerda + \lfloor (direita - esquerda) / 2 \rfloor$
            \IF{$ x = A[meio]$}
                \RETURN $ i $
            \ELSIF{$ x > A[meio] $}
                \STATE $esquerda \leftarrow meio + 1$
            \ELSE
                \STATE $direita \leftarrow meio - 1$
            \ENDIF
        \ENDWHILE
        \RETURN \FALSE
    \end{algorithmic}
    \label{alg:busca_binaria}
\end{algorithm}

\begin{table}[h]
    \centering
    \resizebox{\textwidth}{!}{
        \input{quadro_algoritmos}
    }
    \caption{Quadro comparativo dos algoritmos de busca.}
    \small{Fonte: Elaborado com base em \cite{sultana_brief_2017}.}
    \label{quadro:comparativo_buscas}
\end{table}
\footnotetext{Custos para busca por salto simples, quando o custo do salto é igual ao custo da busca sequencial. \label{jump_search_simple}}

\section{Materiais e Métodos}

Os algoritmos da Seção \ref{sec:algoritmos}, foram implementados\footnote{Código e outras informações disponíveis em https://github.com/rsn86/edaa-01} na linguagem Python e comparados empiricamente na busca em arranjos estáticos, de valores inteiros sem repetições, com tamanhos entre 100.000 e 1.000.000 de elementos, em intervalos de 100.000.

Para cada tamanho de arranjo, foram gerados 2 cenários de testes:
\begin{itemize}
    \item Aleatório: Arranjo e chave de busca gerados aleatoriamente\footnote{Valores entre 0 e 100 vezes o tamanho do arranjo.}. Para cada arranjo foram executadas 100 buscas com chaves distintas.
    \item Pior caso: Chave de busca selecionada para maximizar o número de comparações de cada algoritmo\footnote{Arranjo do cenário aleatório, ordenado crescentemente, e chave de busca o último elemento do arranjo.}. Foram executadas 3 buscas.
\end{itemize}

A cada execução, registrou-se o número de comparações ($a_{i} = x$) e o tempo de execução (em ms) de cada um dos três métodos de busca implementados. Foram calculadas as médias ($\bar{x}$) e os desvios padrão ($\sigma$).

Além disso, para o cenário aleatório, também foi registrado o percentual de vezes em que a chave estava presente no arranjo, coluna $x \in A$ da Tabela \ref{tab:resultados}.

Para os métodos de busca que necessitam de arranjos ordenados, o tempo de ordenação\footnote{Utilizando o método de ordenação padrão da linguagem Python - TimSort.} foi registrado separadamente do tempo de execução das buscas\footnote{Calculou-se a média e desvio padrão de 3 execuções}.

Os testes foram conduzidos em um ambiente de desenvolvimento colaborativo\footnote{https://research.google.com/colaboratory/faq.html}, cujos recursos são compartilhados e estão sujeitos à flutuações de disponibilidade. Para minimizar o impacto dessas flutuações nos resultados, foram realizadas 2 execuções adicionais para cada teste, e os 2 maiores tempos de execução foram descartados.

\section{Resultados}
A Tabela \ref{tab:resultados} sumariza os resultados dos testes realizados e está dividida em 3 conjuntos de colunas: cenários aleatório e pior caso, e ordenação.

Para a ordenação são apresentados apenas os tempos médios e os desvios padrão, uma vez que se utilizou a implementação disponibilizada pela linguagem de programação e não se tem acesso aos números de comparações e trocas realizadas.

Para o cenário de pior caso, apresentou-se o número de comparações (Comp. $x$), e não a média e desvio padrão, pois em todas as execuções foram realizadas o máximo de comparações de cada método, resultando num valor fixo e portanto $\sigma = 0$.

Já no cenário aleatório, a coluna $x \in A$, traz apenas a média das vezes em que a chave esta presente no arranjo, o desvio padrão é irrelevante para análise em questão.

\begin{table}[h]
    \centering
    \resizebox{\textwidth}{!}{
        \begin{table}
\caption{Comparação empírica do desempenho dos métodos de busca}
\label{tab:resultados}
\begin{tabular}{llrrrrrrrrrr}
\toprule
 &  & \multicolumn{5}{r}{Aleatório} & \multicolumn{3}{r}{Pior Caso} & \multicolumn{2}{r}{Ordenação} \\
 \cmidrule(lr){3-7} \cmidrule(lr){8-10} \cmidrule(lr){11-12}
 &  & \multicolumn{2}{r}{Tempo (ms)} & \multicolumn{2}{r}{Comp.} & $x \in A$ & \multicolumn{2}{r}{Tempo (ms)} & Comp. & \multicolumn{2}{r}{Tempo (ms)} \\
 \cmidrule(lr){3-4} \cmidrule(lr){5-6} \cmidrule(lr){7-7} \cmidrule(lr){8-9} \cmidrule(lr){10-10} \cmidrule(lr){11-12}
 Tam. & Busca & $\bar{x}$ & $\sigma$ & $\bar{x}$ & $\sigma$ & $\bar{x}$ & $\bar{x}$ & $\sigma$ & $x$ & $\bar{x}$ & $\sigma$ \\
\midrule
\multirow[c]{3}{*}{100.000} & Binária & 0,02 & 0,00 & 15,98 & 1,33 & 0,57 & 0,01 & 0,00 & 17 & \multirow[r]{3}{*}{31,76} & \multirow[r]{3}{*}{0,28} \\
 & Por saltos & 0,16 & 0,07 & 382,17 & 131,61 & 0,59 & 0,19 & 0,01 & 633 \\
 & Sequencial & 22,17 & 10,35 & 71.065,25 & 34.653,15 & 0,59 & 25,97 & 0,51 & 100.000 \\
\cline{1-12}
\multirow[c]{3}{*}{200.000} & Binária & 0,02 & 0,00 & 17,31 & 0,99 & 0,48 & 0,02 & 0,00 & 18 & \multirow[r]{3}{*}{111,73} & \multirow[r]{3}{*}{3,95} \\
 & Por saltos & 0,21 & 0,09 & 556,72 & 175,55 & 0,46 & 0,33 & 0,01 & 895 \\
 & Sequencial & 49,25 & 18,93 & 150.370,68 & 66.744,42 & 0,48 & 58,95 & 0,98 & 200.000 \\
\cline{1-12}
\multirow[c]{3}{*}{300.000} & Binária & 0,02 & 0,00 & 17,68 & 1,53 & 0,60 & 0,02 & 0,00 & 19 & \multirow[r]{3}{*}{118,69} & \multirow[r]{3}{*}{1,59} \\
 & Por saltos & 0,27 & 0,13 & 637,97 & 252,79 & 0,60 & 0,60 & 0,14 & 1.096 \\
 & Sequencial & 74,57 & 33,19 & 201.535,38 & 102.630,73 & 0,61 & 108,03 & 13,77 & 300.000 \\
\cline{1-12}
\multirow[c]{3}{*}{400.000} & Binária & 0,02 & 0,00 & 18,05 & 1,34 & 0,50 & 0,02 & 0,01 & 19 & \multirow[r]{3}{*}{167,89} & \multirow[r]{3}{*}{7,89} \\
 & Por saltos & 0,34 & 0,14 & 794,77 & 247,17 & 0,50 & 0,69 & 0,15 & 1.265 \\
 & Sequencial & 105,87 & 36,56 & 302.179,91 & 126.676,43 & 0,50 & 154,11 & 14,50 & 400.000 \\
\cline{1-12}
\multirow[c]{3}{*}{500.000} & Binária & 0,02 & 0,00 & 18,48 & 1,06 & 0,48 & 0,02 & 0,00 & 19 & \multirow[r]{3}{*}{209,18} & \multirow[r]{3}{*}{4,45} \\
 & Por saltos & 0,38 & 0,16 & 908,12 & 283,17 & 0,50 & 0,54 & 0,02 & 1.415 \\
 & Sequencial & 139,39 & 44,74 & 390.926,05 & 151.481,00 & 0,50 & 158,79 & 2,41 & 500.000 \\
\cline{1-12}
\multirow[c]{3}{*}{600.000} & Binária & 0,02 & 0,00 & 18,86 & 1,19 & 0,42 & 0,02 & 0,00 & 20 & \multirow[r]{3}{*}{270,69} & \multirow[r]{3}{*}{6,39} \\
 & Por saltos & 0,38 & 0,16 & 938,88 & 325,90 & 0,42 & 0,60 & 0,02 & 1.550 \\
 & Sequencial & 168,52 & 56,69 & 468.179,95 & 190.295,27 & 0,42 & 197,24 & 3,36 & 600.000 \\
\cline{1-12}
\multirow[c]{3}{*}{700.000} & Binária & 0,02 & 0,00 & 18,98 & 1,23 & 0,48 & 0,02 & 0,00 & 20 & \multirow[r]{3}{*}{467,85} & \multirow[r]{3}{*}{29,30} \\
 & Por saltos & 0,44 & 0,19 & 1.033,14 & 386,35 & 0,48 & 0,66 & 0,01 & 1.674 \\
 & Sequencial & 191,76 & 62,69 & 532.646,87 & 218.858,37 & 0,48 & 232,37 & 6,09 & 700.000 \\
\cline{1-12}
\multirow[c]{3}{*}{800.000} & Binária & 0,02 & 0,00 & 19,28 & 1,04 & 0,52 & 0,02 & 0,01 & 20 & \multirow[r]{3}{*}{381,68} & \multirow[r]{3}{*}{22,68} \\
 & Por saltos & 0,45 & 0,18 & 1.089,20 & 403,75 & 0,53 & 0,99 & 0,22 & 1.789 \\
 & Sequencial & 211,25 & 76,47 & 568.020,71 & 268.519,64 & 0,53 & 314,76 & 34,61 & 800.000 \\
\cline{1-12}
\multirow[c]{3}{*}{900.000} & Binária & 0,02 & 0,00 & 19,52 & 0,83 & 0,50 & 0,02 & 0,00 & 20 & \multirow[r]{3}{*}{568,04} & \multirow[r]{3}{*}{112,58} \\
 & Por saltos & 0,52 & 0,24 & 1.224,75 & 456,39 & 0,50 & 0,78 & 0,00 & 1.898 \\
 & Sequencial & 253,15 & 88,47 & 683.849,38 & 306.246,35 & 0,50 & 299,04 & 5,37 & 900.000 \\
\cline{1-12}
\multirow[c]{3}{*}{1.000.000} & Binária & 0,02 & 0,00 & 19,55 & 1,00 & 0,55 & 0,02 & 0,00 & 20 & \multirow[r]{3}{*}{496,31} & \multirow[r]{3}{*}{22,11} \\
 & Por saltos & 0,52 & 0,26 & 1.223,49 & 437,24 & 0,55 & 0,98 & 0,29 & 2.000 \\
 & Sequencial & 270,34 & 100,09 & 710.240,35 & 349.607,52 & 0,55 & 414,40 & 64,77 & 1.000.000 \\
\bottomrule
\end{tabular}
\end{table}

    }
    \caption{Comparação empírica do desempenho dos métodos de busca}
    %\small{Fonte: Do autor}
    \label{tab:resultados}
\end{table} 

A influência da fluatação na disponibilidade dos recursos pode ser observada ao comparar os valores dos tempos de ordenação no cenário pior caso. Sendo o arranjo, a chave de busca e o número de comparações idênticos entre as execuções, esperava-se que os desvios padrão fossem pequenos, próximos a zero. Porém, apresentaram valores absolutos e relativos elevados, chegando a 64,77 ms e CV\footnote{Coeficiente de variação: $CV=100\frac{\sigma}{\bar{x}}$} de 15,63\%, para 1 milhão de elementos. Ainda, pode ser facilmente observada, neste mesmo cenário, pelos menores tempos de busca em conjuntos maiores, por ex. nos tempos para 800 e 900 mil elementos.

O método de ordenação utilizado, beneficia-se da existência de sequências ordenadas\footnote{Máximas sequências monotônicas - sequências de maior comprimento sem inversão na ordenação.} no conjunto de dados. Isso pode explicar, pelo menos parcialmente, a aparente incongruência nos tempos de ordenação, nos quais arranjos maiores, com 800 mil e 1 milhão de elementos, apresentaram tempos menores que os arranjos de 700 mil e 900 mil elementos. Conforme \cite{auger:hal-01798381}, o custo do TimSort é $O(n + n\mathcal{H})$, a entropia $\mathcal{H}$\footnote{$\mathcal{H}=-\sum{\frac{r_{i}}{n} log_{2}(\frac{r_{i}}{n})}$, $r_{i} - $ comprimento da i-ésima sequência monotônica máxima.} depende do número e comprimento das sequencias parcialmente ordenadas. Porém, seu cálculo não é trivial e foge do escopo deste trabalho, sendo assim, $\mathcal{H}$ não foi calculada.

Os custos encontrados para o pior caso, Tabela\ref{tab:resultados}, correspondem aos descritos na literatura e sumarizados no quadro da Tabela \ref{quadro:comparativo_buscas}, divergindo apenas para o algoritmo de buscas por saltos, sugerindo que o custo do salto é diferente do custo da busca sequencial\footnote{Nesse caso o tamanho do salto deveria ser $\sqrt{(a/b)N}$, com custo $\sqrt{abN}$. Custos, a: salto, b: busca sequencial. Porém o tamanho do salto utilizado foi $\sqrt{N}$.} ou que a literatura não considera o custo da busca sequencial, uma vez que o custo encontrado ($2\sqrt{N}$), corresponde ao custo reportado na literatura ($\sqrt{N}$), acrescido do custo da busca sequencial em um bloco de tamanho $\sqrt{N}$.

\section{Conclusões}

Conhecendo-se o cenário de utilização, é possível selecionar algoritmos mais adequados e performáticos. \cite{cappelle_searching_2021} apresenta um interessante levantamento sobre diversos algoritmos de busca em vetores. Nesse sentido, a principal conclusão deste estudo é que o custo de ordenação justifica-se para os casos de múltiplas buscas sobre o mesmo arranjo. E, apesar do algoritmo de busca binária apresentar o melhor desempenho, tanto em termos de tempo quanto comparações, ele não é adequado para vetores com muitas inserções e remoções, devido ao custo associado em se manter o vetor ordenado. Nesse caso, pode-se considerar a utilização de árvores de busca binária, preferencialmente as balanceadas, estruturas de dados não lineares, cujas restrições asseguram propriedades interessantes, garantindo bons custos de inserção, remoção e busca.

O fato de ter sido adotado um arranjo com valores espalhados, não contínuos, e chaves de buscas que podiam estar ausentes do conjunto, apesar de contribuir para o aumento nos tempos de execução e no número de comparações,
aproxima o cenário de testes de um cenário de uso real, no qual não se pode controlar a distribuição dos valores e não se tem certeza sobre a presença da chave de busca no conjunto.

A influência da entropia carece de investigação. Alternativamente, poderia ser adotado um método menos suscetível a tal fator.

A execução em ambiente compartilhado permitiu visualizar as relações de grandeza entre os métodos de busca e os tamanhos dos arranjos, porém valores mais precisos poderiam ser obtidos em um ambiente controlado, com alocação dedicada de recursos.

A degradação na linearidade dos tempos de busca, observável no cenário pior caso da busca sequencial, com o crescimento do tamanho dos arranjos, pode estar relacionada à hierarquia de memória e a necessidade de movimentação dos dados. Um aprofundamento do estudo envolveria maiores conhecimentos sobre a arquitetura do ambiente, como modelo da CPU, tamanho e temporização das memórias cache e RAM.

\bibliographystyle{sbc}
\bibliography{edaa01}

\end{document}
